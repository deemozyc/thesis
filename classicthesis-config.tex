% ****************************************************************************************************
% classicthesis-config.tex 
% formerly known as loadpackages.sty, classicthesis-ldpkg.sty, and classicthesis-preamble.sty 
% Use it at the beginning of your ClassicThesis.tex, or as a LaTeX Preamble 
% in your ClassicThesis.{tex,lyx} with % ****************************************************************************************************
% classicthesis-config.tex 
% formerly known as loadpackages.sty, classicthesis-ldpkg.sty, and classicthesis-preamble.sty 
% Use it at the beginning of your ClassicThesis.tex, or as a LaTeX Preamble 
% in your ClassicThesis.{tex,lyx} with % ****************************************************************************************************
% classicthesis-config.tex 
% formerly known as loadpackages.sty, classicthesis-ldpkg.sty, and classicthesis-preamble.sty 
% Use it at the beginning of your ClassicThesis.tex, or as a LaTeX Preamble 
% in your ClassicThesis.{tex,lyx} with % ****************************************************************************************************
% classicthesis-config.tex 
% formerly known as loadpackages.sty, classicthesis-ldpkg.sty, and classicthesis-preamble.sty 
% Use it at the beginning of your ClassicThesis.tex, or as a LaTeX Preamble 
% in your ClassicThesis.{tex,lyx} with \input{classicthesis-config}
% ****************************************************************************************************  
% If you like the classicthesis, then I would appreciate a postcard. 
% My address can be found in the file ClassicThesis.pdf. A collection 
% of the postcards I received so far is available online at 
% http://postcards.miede.de
% ****************************************************************************************************


% ****************************************************************************************************
% 0. Set the encoding of your files. UTF-8 is the only sensible encoding nowadays. If you can't read
% äöüßáéçèê∂åëæƒÏ€ then change the encoding setting in your editor, not the line below. If your editor
% does not support utf8 use another editor!
% ****************************************************************************************************
\PassOptionsToPackage{usenames,dvipsnames}{xcolor}%needed because several packages include xcolor
\PassOptionsToPackage{utf8}{inputenc}
\usepackage{inputenc}
        
% ****************************************************************************************************
% 1. Configure classicthesis for your needs here, e.g., remove "drafting" below 
% in order to deactivate the time-stamp on the pages
% ****************************************************************************************************

% \PassOptionsToPackage{eulerchapternumbers,listings,%drafting,%
%					 pdfspacing,%floatperchapter,%linedheaders,%
%					 subfig,beramono,eulermath,parts}{arsclassica}

\PassOptionsToPackage{
  drafting=false,    % print version information on the bottom of the pages
  tocaligned=false, % the left column of the toc will be aligned (no indentation)
  dottedtoc=false,  % page numbers in ToC flushed right
  eulerchapternumbers=true, % use AMS Euler for chapter font (otherwise Palatino)
  linedheaders=false,       % chaper headers will have line above and beneath
  floatperchapter=true,     % numbering per chapter for all floats (i.e., Figure 1.1)
  minionprospacing=true,
  eulermath=true,  % use awesome Euler fonts for mathematical formulae (only with pdfLaTeX)
  beramono=true,    % toggle a nice monospaced font (w/ bold)
  palatino=true,    % deactivate standard font for loading another one, see the last section at the end of this file for suggestions
  style=arsclassica % classicthesis, arsclassica
}{classicthesis}


% ********************************************************************
% Available options for classicthesis.sty 
% (see ClassicThesis.pdf for more information):
% drafting
% parts nochapters linedheaders
% eulerchapternumbers beramono eulermath pdfspacing minionprospacing
% tocaligned dottedtoc manychapters
% listings floatperchapter subfig
% ********************************************************************

\usepackage{url}
\usepackage{nameref}
\usepackage{epigraph} %used for neat epigraphs at begging of chapters
\usepackage{verse}
\usepackage{datetime} %needed to add the date just below
\usepackage{pdflscape} %used to rotate pages with wide tables
\usepackage{rotating} %used to rotate wide figures
\usepackage{siunitx} %used for nice number formatting
\usepackage{pgf} % needed for arrays among other things
% ****************************************************************************************************
% 2. Personal data and user ad-hoc commands
% ****************************************************************************************************

\renewcommand{\epigraphsize}{\scriptsize}

\def\authors{
  {Student No 1/Student ID 1},
  {Student No 2/Student ID 2},
  {Student No 3/Student ID 3},
  {Student No 4/Student ID 4}}
\def\myTitle{Experimental CS Thesis Title\xspace}
\def\mySubtitle{A Guide for Content \& Style\xspace}
\def\myDegree{Master's Thesis\xspace}
\def\myShortNames{Student 1, 2, 3 \& 4\xspace}
\def\myGroup{My Group (if applicable)}
%\newcommand{\myStudentId}{20051234\xspace}
%\newcommand{\myNameToo}{Put Other Student Name Here\xspace}
%\newcommand{\myStudentIdToo}{20051235\xspace}
\def\myProf{Niels Olof Bouvin\xspace}
\def\myOtherProf{Put name here\xspace}
\def\mySupervisor{Put name here\xspace}
\def\myFaculty{Faculty of Natural Sciences\xspace}
\def\myDepartment{Department of Computer Science\xspace}
\def\myUni{Aarhus University\xspace}
\def\myLocation{Aarhus\xspace}
\def\myTime{\monthname\ \the\year\xspace}
\def\myVersion{\xspace}


% ********************************************************************
% Setup, finetuning, and useful commands
% ********************************************************************
\newcounter{dummy} % necessary for correct hyperlinks (to index, bib, etc.)
\newlength{\abcd} % for ab..z string length calculation
\providecommand{\mLyX}{L\kern-.1667em\lower.25em\hbox{Y}\kern-.125emX\@}
\providecommand{\mBibTeX}{\textsc{Bib}\negthinspace\TeX\@}
% ****************************************************************************************************


% ****************************************************************************************************
% 3. Loading some handy packages
% ****************************************************************************************************
% ******************************************************************** 
% Packages with options that might require adjustments
% ******************************************************************** 
%\PassOptionsToPackage{ngerman,american}{babel}   % change this to your language(s)
% Spanish languages need extra options in order to work with this template
%\PassOptionsToPackage{spanish,es-lcroman}{babel}
\usepackage[american]{babel}                  
\usepackage{microtype}
\usepackage{csquotes}
\usepackage{ragged2e}
% \usepackage[usenames,dvipsnames]{xcolor}
\usepackage{xcolor}
\usepackage{pgf-umlsd}
\usepackage{pgfplots}
\usepackage{pgfplotstable}
% recommended:
\usepackage{booktabs}
\usepackage{array}
\usepackage{colortbl}
\usepackage{amsthm}

% \usepackage{color}
\definecolor{aublue}{HTML}{193b77} % the dark blue colour used by AU

% Modify the biblabel format to use aublue for the source attribution
\renewcommand{\mkcitation}[1]{\xspace\textcolor{aublue}{[#1]}}




%\usepackage[square,numbers,sort&compress]{natbib}
%\bibliographystyle{abbrvnat}

\PassOptionsToPackage{%
   %backend=biber, %instead of bibtex
	backend=bibtex8,bibencoding=ascii,%
	language=auto,%
	style=numeric-comp,%
   %style=authoryear-comp, % Author 1999, 2010
   %bibstyle=authoryear,dashed=false, % dashed: substitute rep. author with ---
   sorting=nyt, % name, year, title
   maxbibnames=10, % default: 3, et al.
   %backref=true,%
   natbib=true % natbib compatibility mode (\citep and \citet still work)
}{biblatex}
\usepackage{biblatex}

\PassOptionsToPackage{fleqn}{amsmath}       % math environments and more by the AMS 
\usepackage{amsmath}
%
% ******************************************************************** 
% General useful packages
% ******************************************************************** 
\PassOptionsToPackage{T1}{fontenc} % T2A for cyrillics
\usepackage{fontenc}   
\usepackage{comment} 
\usepackage{textcomp} % fix warning with missing font shapes
\usepackage{scrhack} % fix warnings when using KOMA with listings package          
\usepackage{xspace} % to get the spacing after macros right  
\newcommand{\eg}{e.g.,\xspace}
\newcommand{\Eg}{E.g.,\xspace}
\newcommand{\etal}{et~al.\xspace}
\newcommand{\etc}{etc.\@\xspace}
\newcommand{\ie}{i.e.,\xspace}
\newcommand{\viz}{viz.\@\xspace}

\usepackage{mparhack} % get marginpar right
\usepackage{fixltx2e} % fixes some LaTeX stuff --> since 2015 in the LaTeX kernel (see below)
%\usepackage[latest]{latexrelease} % will be used once available in more distributions (ISSUE #107)
\PassOptionsToPackage{printonlyused,smaller}{acronym} 
    \usepackage{acronym} % nice macros for handling all acronyms in the thesis
    %\renewcommand{\bflabel}[1]{{#1}\hfill} % fix the list of acronyms --> no longer working
    %\renewcommand*{\acsfont}[1]{\textsc{#1}} 
    \renewcommand*{\aclabelfont}[1]{\acsfont{#1}}
% ****************************************************************************************************


% ****************************************************************************************************
% 4. Setup floats: tables, (sub)figures, and captions
% ****************************************************************************************************
\usepackage{tabularx} % better tables
    \setlength{\extrarowheight}{3pt} % increase table row height
\newcommand{\tableheadline}[1]{\multicolumn{1}{c}{\spacedlowsmallcaps{#1}}}
\newcommand{\myfloatalign}{\centering} % to be used with each float for alignment
\usepackage{caption}
% Thanks to cgnieder and Claus Lahiri
% http://tex.stackexchange.com/questions/69349/spacedlowsmallcaps-in-caption-label
% [REMOVED DUE TO OTHER PROBLEMS, SEE ISSUE #82]    
%\DeclareCaptionLabelFormat{smallcaps}{\bothIfFirst{#1}{~}\MakeTextLowercase{\textsc{#2}}}
%\captionsetup{font=small,labelformat=smallcaps} % format=hang,
\captionsetup{font=small} % format=hang,
\usepackage{subfig}  
% ****************************************************************************************************


% ****************************************************************************************************
% 5. Setup code listings
% ****************************************************************************************************
\usepackage{listings}
%\lstset{emph={trueIndex,root},emphstyle=\color{BlueViolet}}%\underbar} % for special keywords
\lstset{language=[LaTeX]Tex,%C++,
    keywordstyle=\bfseries,%\color{RoyalBlue},%\bfseries,
    basicstyle=\footnotesize\ttfamily,
    identifierstyle=\color{RoyalBlue},
    commentstyle=\color{Green}\itshape,
    stringstyle=\sffamily,
    numbers=left,
    numberstyle=\tiny,
    stepnumber=1,
    numbersep=8pt,
    showstringspaces=false,
    breaklines=true,
    frameround=ffft,
    belowcaptionskip=.75\baselineskip,
    frame=single,
    rulecolor=\color{aublue}
}
% ****************************************************************************************************             


% ****************************************************************************************************
% 6. PDFLaTeX, hyperreferences and citation backreferences
% ****************************************************************************************************
% ********************************************************************
% Using PDFLaTeX
% ********************************************************************
\PassOptionsToPackage{pdftex,hyperfootnotes=false,pdfpagelabels}{hyperref}
    \usepackage{hyperref}  % backref linktocpage pagebackref
\pdfcompresslevel=9
\pdfadjustspacing=1 
\PassOptionsToPackage{pdftex}{graphicx}
    \usepackage{graphicx} 
 

% ********************************************************************
% Hyperreferences
% ********************************************************************
\hypersetup{%
    %draft, % = no hyperlinking at all (useful in b/w printouts)
    colorlinks=true, linktocpage=true, pdfstartpage=3, pdfstartview=FitV,%
    % uncomment the following line if you want to have black links (e.g., for printing)
    %colorlinks=false, linktocpage=false, pdfstartpage=3, pdfstartview=FitV, pdfborder={0 0 0},%
    breaklinks=true, pdfpagemode=UseNone, pageanchor=true, pdfpagemode=UseOutlines,%
    plainpages=false, bookmarksnumbered, bookmarksopen=true, bookmarksopenlevel=1,%
    hypertexnames=true, pdfhighlight=/O,%nesting=true,%frenchlinks,%
    urlcolor=webbrown, linkcolor=RoyalBlue, citecolor=webgreen, %pagecolor=RoyalBlue,%
    %urlcolor=Black, linkcolor=Black, citecolor=Black, %pagecolor=Black,%
    pdftitle={\myTitle},%
    pdfauthor={\textcopyright\ \myShortNames, \myUni, \myFaculty},%
    pdfsubject={},%
    pdfkeywords={},%
    pdfcreator={pdfLaTeX},%
    pdfproducer={LaTeX with hyperref and classicthesis}%
}   

% ********************************************************************
% Setup autoreferences
% ********************************************************************
% There are some issues regarding autorefnames
% http://www.ureader.de/msg/136221647.aspx
% http://www.tex.ac.uk/cgi-bin/texfaq2html?label=latexwords
% you have to redefine the makros for the 
% language you use, e.g., american, ngerman
% (as chosen when loading babel/AtBeginDocument)
% ********************************************************************
\makeatletter
\@ifpackageloaded{babel}%
    {%
       \addto\extrasamerican{%
			\renewcommand*{\figureautorefname}{Figure}%
			\renewcommand*{\tableautorefname}{Table}%
			\renewcommand*{\partautorefname}{Part}%
			\renewcommand*{\chapterautorefname}{Chapter}%
			\renewcommand*{\sectionautorefname}{Section}%
			\renewcommand*{\subsectionautorefname}{Section}%
			\renewcommand*{\subsubsectionautorefname}{Section}%     
                }%
       \addto\extrasngerman{% 
			\renewcommand*{\paragraphautorefname}{Absatz}%
			\renewcommand*{\subparagraphautorefname}{Unterabsatz}%
			\renewcommand*{\footnoteautorefname}{Fu\"snote}%
			\renewcommand*{\FancyVerbLineautorefname}{Zeile}%
			\renewcommand*{\theoremautorefname}{Theorem}%
			\renewcommand*{\appendixautorefname}{Anhang}%
			\renewcommand*{\equationautorefname}{Gleichung}%        
			\renewcommand*{\itemautorefname}{Punkt}%
                }%  
            % Fix to getting autorefs for subfigures right (thanks to Belinda Vogt for changing the definition)
            \providecommand{\subfigureautorefname}{\figureautorefname}%             
    }{\relax}
\makeatother


% ****************************************************************************************************
% 7. Last calls before the bar closes
% ****************************************************************************************************
% ********************************************************************
% Development Stuff
% ********************************************************************
\listfiles
%\PassOptionsToPackage{l2tabu,orthodox,abort}{nag}
%   \usepackage{nag}
%\PassOptionsToPackage{warning, all}{onlyamsmath}
%   \usepackage{onlyamsmath}

% ********************************************************************
% Last, but not least...
% ********************************************************************
\usepackage{classicthesis} 
% ****************************************************************************************************


% ****************************************************************************************************
% 8. Further adjustments (experimental)
% ****************************************************************************************************
% ********************************************************************
% Changing the text area
% ********************************************************************
%\linespread{1.05} % a bit more for Palatino
%\areaset[current]{312pt}{761pt} % 686 (factor 2.2) + 33 head + 42 head \the\footskip
%\setlength{\marginparwidth}{7em}%
%\setlength{\marginparsep}{2em}%

% ********************************************************************
% Using different fonts
% ********************************************************************
%\usepackage[oldstylenums]{kpfonts} % oldstyle notextcomp
%\usepackage[osf]{libertine}
%\usepackage[light,condensed,math]{iwona}
%\renewcommand{\sfdefault}{iwona}
%\usepackage{lmodern} % <-- no osf support :-(
%\usepackage{cfr-lm} % 
%\usepackage[urw-garamond]{mathdesign} <-- no osf support :-(
%\usepackage[default,osfigures]{opensans} % scale=0.95 
%\usepackage[sfdefault]{FiraSans}
% ****************************************************************************************************



%%% Local Variables:
%%% mode: latex
%%% TeX-master: "./ClassicThesis"
%%% End:

% ****************************************************************************************************  
% If you like the classicthesis, then I would appreciate a postcard. 
% My address can be found in the file ClassicThesis.pdf. A collection 
% of the postcards I received so far is available online at 
% http://postcards.miede.de
% ****************************************************************************************************


% ****************************************************************************************************
% 0. Set the encoding of your files. UTF-8 is the only sensible encoding nowadays. If you can't read
% äöüßáéçèê∂åëæƒÏ€ then change the encoding setting in your editor, not the line below. If your editor
% does not support utf8 use another editor!
% ****************************************************************************************************
\PassOptionsToPackage{usenames,dvipsnames}{xcolor}%needed because several packages include xcolor
\PassOptionsToPackage{utf8}{inputenc}
\usepackage{inputenc}
        
% ****************************************************************************************************
% 1. Configure classicthesis for your needs here, e.g., remove "drafting" below 
% in order to deactivate the time-stamp on the pages
% ****************************************************************************************************

% \PassOptionsToPackage{eulerchapternumbers,listings,%drafting,%
%					 pdfspacing,%floatperchapter,%linedheaders,%
%					 subfig,beramono,eulermath,parts}{arsclassica}

\PassOptionsToPackage{
  drafting=false,    % print version information on the bottom of the pages
  tocaligned=false, % the left column of the toc will be aligned (no indentation)
  dottedtoc=false,  % page numbers in ToC flushed right
  eulerchapternumbers=true, % use AMS Euler for chapter font (otherwise Palatino)
  linedheaders=false,       % chaper headers will have line above and beneath
  floatperchapter=true,     % numbering per chapter for all floats (i.e., Figure 1.1)
  minionprospacing=true,
  eulermath=true,  % use awesome Euler fonts for mathematical formulae (only with pdfLaTeX)
  beramono=true,    % toggle a nice monospaced font (w/ bold)
  palatino=true,    % deactivate standard font for loading another one, see the last section at the end of this file for suggestions
  style=arsclassica % classicthesis, arsclassica
}{classicthesis}


% ********************************************************************
% Available options for classicthesis.sty 
% (see ClassicThesis.pdf for more information):
% drafting
% parts nochapters linedheaders
% eulerchapternumbers beramono eulermath pdfspacing minionprospacing
% tocaligned dottedtoc manychapters
% listings floatperchapter subfig
% ********************************************************************

\usepackage{url}
\usepackage{nameref}
\usepackage{epigraph} %used for neat epigraphs at begging of chapters
\usepackage{verse}
\usepackage{datetime} %needed to add the date just below
\usepackage{pdflscape} %used to rotate pages with wide tables
\usepackage{rotating} %used to rotate wide figures
\usepackage{siunitx} %used for nice number formatting
\usepackage{pgf} % needed for arrays among other things
% ****************************************************************************************************
% 2. Personal data and user ad-hoc commands
% ****************************************************************************************************

\renewcommand{\epigraphsize}{\scriptsize}

\def\authors{
  {Student No 1/Student ID 1},
  {Student No 2/Student ID 2},
  {Student No 3/Student ID 3},
  {Student No 4/Student ID 4}}
\def\myTitle{Experimental CS Thesis Title\xspace}
\def\mySubtitle{A Guide for Content \& Style\xspace}
\def\myDegree{Master's Thesis\xspace}
\def\myShortNames{Student 1, 2, 3 \& 4\xspace}
\def\myGroup{My Group (if applicable)}
%\newcommand{\myStudentId}{20051234\xspace}
%\newcommand{\myNameToo}{Put Other Student Name Here\xspace}
%\newcommand{\myStudentIdToo}{20051235\xspace}
\def\myProf{Niels Olof Bouvin\xspace}
\def\myOtherProf{Put name here\xspace}
\def\mySupervisor{Put name here\xspace}
\def\myFaculty{Faculty of Natural Sciences\xspace}
\def\myDepartment{Department of Computer Science\xspace}
\def\myUni{Aarhus University\xspace}
\def\myLocation{Aarhus\xspace}
\def\myTime{\monthname\ \the\year\xspace}
\def\myVersion{\xspace}


% ********************************************************************
% Setup, finetuning, and useful commands
% ********************************************************************
\newcounter{dummy} % necessary for correct hyperlinks (to index, bib, etc.)
\newlength{\abcd} % for ab..z string length calculation
\providecommand{\mLyX}{L\kern-.1667em\lower.25em\hbox{Y}\kern-.125emX\@}
\providecommand{\mBibTeX}{\textsc{Bib}\negthinspace\TeX\@}
% ****************************************************************************************************


% ****************************************************************************************************
% 3. Loading some handy packages
% ****************************************************************************************************
% ******************************************************************** 
% Packages with options that might require adjustments
% ******************************************************************** 
%\PassOptionsToPackage{ngerman,american}{babel}   % change this to your language(s)
% Spanish languages need extra options in order to work with this template
%\PassOptionsToPackage{spanish,es-lcroman}{babel}
\usepackage[american]{babel}                  
\usepackage{microtype}
\usepackage{csquotes}
\usepackage{ragged2e}
% \usepackage[usenames,dvipsnames]{xcolor}
\usepackage{xcolor}
\usepackage{pgf-umlsd}
\usepackage{pgfplots}
\usepackage{pgfplotstable}
% recommended:
\usepackage{booktabs}
\usepackage{array}
\usepackage{colortbl}
\usepackage{amsthm}

% \usepackage{color}
\definecolor{aublue}{HTML}{193b77} % the dark blue colour used by AU

% Modify the biblabel format to use aublue for the source attribution
\renewcommand{\mkcitation}[1]{\xspace\textcolor{aublue}{[#1]}}




%\usepackage[square,numbers,sort&compress]{natbib}
%\bibliographystyle{abbrvnat}

\PassOptionsToPackage{%
   %backend=biber, %instead of bibtex
	backend=bibtex8,bibencoding=ascii,%
	language=auto,%
	style=numeric-comp,%
   %style=authoryear-comp, % Author 1999, 2010
   %bibstyle=authoryear,dashed=false, % dashed: substitute rep. author with ---
   sorting=nyt, % name, year, title
   maxbibnames=10, % default: 3, et al.
   %backref=true,%
   natbib=true % natbib compatibility mode (\citep and \citet still work)
}{biblatex}
\usepackage{biblatex}

\PassOptionsToPackage{fleqn}{amsmath}       % math environments and more by the AMS 
\usepackage{amsmath}
%
% ******************************************************************** 
% General useful packages
% ******************************************************************** 
\PassOptionsToPackage{T1}{fontenc} % T2A for cyrillics
\usepackage{fontenc}   
\usepackage{comment} 
\usepackage{textcomp} % fix warning with missing font shapes
\usepackage{scrhack} % fix warnings when using KOMA with listings package          
\usepackage{xspace} % to get the spacing after macros right  
\newcommand{\eg}{e.g.,\xspace}
\newcommand{\Eg}{E.g.,\xspace}
\newcommand{\etal}{et~al.\xspace}
\newcommand{\etc}{etc.\@\xspace}
\newcommand{\ie}{i.e.,\xspace}
\newcommand{\viz}{viz.\@\xspace}

\usepackage{mparhack} % get marginpar right
\usepackage{fixltx2e} % fixes some LaTeX stuff --> since 2015 in the LaTeX kernel (see below)
%\usepackage[latest]{latexrelease} % will be used once available in more distributions (ISSUE #107)
\PassOptionsToPackage{printonlyused,smaller}{acronym} 
    \usepackage{acronym} % nice macros for handling all acronyms in the thesis
    %\renewcommand{\bflabel}[1]{{#1}\hfill} % fix the list of acronyms --> no longer working
    %\renewcommand*{\acsfont}[1]{\textsc{#1}} 
    \renewcommand*{\aclabelfont}[1]{\acsfont{#1}}
% ****************************************************************************************************


% ****************************************************************************************************
% 4. Setup floats: tables, (sub)figures, and captions
% ****************************************************************************************************
\usepackage{tabularx} % better tables
    \setlength{\extrarowheight}{3pt} % increase table row height
\newcommand{\tableheadline}[1]{\multicolumn{1}{c}{\spacedlowsmallcaps{#1}}}
\newcommand{\myfloatalign}{\centering} % to be used with each float for alignment
\usepackage{caption}
% Thanks to cgnieder and Claus Lahiri
% http://tex.stackexchange.com/questions/69349/spacedlowsmallcaps-in-caption-label
% [REMOVED DUE TO OTHER PROBLEMS, SEE ISSUE #82]    
%\DeclareCaptionLabelFormat{smallcaps}{\bothIfFirst{#1}{~}\MakeTextLowercase{\textsc{#2}}}
%\captionsetup{font=small,labelformat=smallcaps} % format=hang,
\captionsetup{font=small} % format=hang,
\usepackage{subfig}  
% ****************************************************************************************************


% ****************************************************************************************************
% 5. Setup code listings
% ****************************************************************************************************
\usepackage{listings}
%\lstset{emph={trueIndex,root},emphstyle=\color{BlueViolet}}%\underbar} % for special keywords
\lstset{language=[LaTeX]Tex,%C++,
    keywordstyle=\bfseries,%\color{RoyalBlue},%\bfseries,
    basicstyle=\footnotesize\ttfamily,
    identifierstyle=\color{RoyalBlue},
    commentstyle=\color{Green}\itshape,
    stringstyle=\sffamily,
    numbers=left,
    numberstyle=\tiny,
    stepnumber=1,
    numbersep=8pt,
    showstringspaces=false,
    breaklines=true,
    frameround=ffft,
    belowcaptionskip=.75\baselineskip,
    frame=single,
    rulecolor=\color{aublue}
}
% ****************************************************************************************************             


% ****************************************************************************************************
% 6. PDFLaTeX, hyperreferences and citation backreferences
% ****************************************************************************************************
% ********************************************************************
% Using PDFLaTeX
% ********************************************************************
\PassOptionsToPackage{pdftex,hyperfootnotes=false,pdfpagelabels}{hyperref}
    \usepackage{hyperref}  % backref linktocpage pagebackref
\pdfcompresslevel=9
\pdfadjustspacing=1 
\PassOptionsToPackage{pdftex}{graphicx}
    \usepackage{graphicx} 
 

% ********************************************************************
% Hyperreferences
% ********************************************************************
\hypersetup{%
    %draft, % = no hyperlinking at all (useful in b/w printouts)
    colorlinks=true, linktocpage=true, pdfstartpage=3, pdfstartview=FitV,%
    % uncomment the following line if you want to have black links (e.g., for printing)
    %colorlinks=false, linktocpage=false, pdfstartpage=3, pdfstartview=FitV, pdfborder={0 0 0},%
    breaklinks=true, pdfpagemode=UseNone, pageanchor=true, pdfpagemode=UseOutlines,%
    plainpages=false, bookmarksnumbered, bookmarksopen=true, bookmarksopenlevel=1,%
    hypertexnames=true, pdfhighlight=/O,%nesting=true,%frenchlinks,%
    urlcolor=webbrown, linkcolor=RoyalBlue, citecolor=webgreen, %pagecolor=RoyalBlue,%
    %urlcolor=Black, linkcolor=Black, citecolor=Black, %pagecolor=Black,%
    pdftitle={\myTitle},%
    pdfauthor={\textcopyright\ \myShortNames, \myUni, \myFaculty},%
    pdfsubject={},%
    pdfkeywords={},%
    pdfcreator={pdfLaTeX},%
    pdfproducer={LaTeX with hyperref and classicthesis}%
}   

% ********************************************************************
% Setup autoreferences
% ********************************************************************
% There are some issues regarding autorefnames
% http://www.ureader.de/msg/136221647.aspx
% http://www.tex.ac.uk/cgi-bin/texfaq2html?label=latexwords
% you have to redefine the makros for the 
% language you use, e.g., american, ngerman
% (as chosen when loading babel/AtBeginDocument)
% ********************************************************************
\makeatletter
\@ifpackageloaded{babel}%
    {%
       \addto\extrasamerican{%
			\renewcommand*{\figureautorefname}{Figure}%
			\renewcommand*{\tableautorefname}{Table}%
			\renewcommand*{\partautorefname}{Part}%
			\renewcommand*{\chapterautorefname}{Chapter}%
			\renewcommand*{\sectionautorefname}{Section}%
			\renewcommand*{\subsectionautorefname}{Section}%
			\renewcommand*{\subsubsectionautorefname}{Section}%     
                }%
       \addto\extrasngerman{% 
			\renewcommand*{\paragraphautorefname}{Absatz}%
			\renewcommand*{\subparagraphautorefname}{Unterabsatz}%
			\renewcommand*{\footnoteautorefname}{Fu\"snote}%
			\renewcommand*{\FancyVerbLineautorefname}{Zeile}%
			\renewcommand*{\theoremautorefname}{Theorem}%
			\renewcommand*{\appendixautorefname}{Anhang}%
			\renewcommand*{\equationautorefname}{Gleichung}%        
			\renewcommand*{\itemautorefname}{Punkt}%
                }%  
            % Fix to getting autorefs for subfigures right (thanks to Belinda Vogt for changing the definition)
            \providecommand{\subfigureautorefname}{\figureautorefname}%             
    }{\relax}
\makeatother


% ****************************************************************************************************
% 7. Last calls before the bar closes
% ****************************************************************************************************
% ********************************************************************
% Development Stuff
% ********************************************************************
\listfiles
%\PassOptionsToPackage{l2tabu,orthodox,abort}{nag}
%   \usepackage{nag}
%\PassOptionsToPackage{warning, all}{onlyamsmath}
%   \usepackage{onlyamsmath}

% ********************************************************************
% Last, but not least...
% ********************************************************************
\usepackage{classicthesis} 
% ****************************************************************************************************


% ****************************************************************************************************
% 8. Further adjustments (experimental)
% ****************************************************************************************************
% ********************************************************************
% Changing the text area
% ********************************************************************
%\linespread{1.05} % a bit more for Palatino
%\areaset[current]{312pt}{761pt} % 686 (factor 2.2) + 33 head + 42 head \the\footskip
%\setlength{\marginparwidth}{7em}%
%\setlength{\marginparsep}{2em}%

% ********************************************************************
% Using different fonts
% ********************************************************************
%\usepackage[oldstylenums]{kpfonts} % oldstyle notextcomp
%\usepackage[osf]{libertine}
%\usepackage[light,condensed,math]{iwona}
%\renewcommand{\sfdefault}{iwona}
%\usepackage{lmodern} % <-- no osf support :-(
%\usepackage{cfr-lm} % 
%\usepackage[urw-garamond]{mathdesign} <-- no osf support :-(
%\usepackage[default,osfigures]{opensans} % scale=0.95 
%\usepackage[sfdefault]{FiraSans}
% ****************************************************************************************************



%%% Local Variables:
%%% mode: latex
%%% TeX-master: "./ClassicThesis"
%%% End:

% ****************************************************************************************************  
% If you like the classicthesis, then I would appreciate a postcard. 
% My address can be found in the file ClassicThesis.pdf. A collection 
% of the postcards I received so far is available online at 
% http://postcards.miede.de
% ****************************************************************************************************


% ****************************************************************************************************
% 0. Set the encoding of your files. UTF-8 is the only sensible encoding nowadays. If you can't read
% äöüßáéçèê∂åëæƒÏ€ then change the encoding setting in your editor, not the line below. If your editor
% does not support utf8 use another editor!
% ****************************************************************************************************
\PassOptionsToPackage{usenames,dvipsnames}{xcolor}%needed because several packages include xcolor
\PassOptionsToPackage{utf8}{inputenc}
\usepackage{inputenc}
        
% ****************************************************************************************************
% 1. Configure classicthesis for your needs here, e.g., remove "drafting" below 
% in order to deactivate the time-stamp on the pages
% ****************************************************************************************************

% \PassOptionsToPackage{eulerchapternumbers,listings,%drafting,%
%					 pdfspacing,%floatperchapter,%linedheaders,%
%					 subfig,beramono,eulermath,parts}{arsclassica}

\PassOptionsToPackage{
  drafting=false,    % print version information on the bottom of the pages
  tocaligned=false, % the left column of the toc will be aligned (no indentation)
  dottedtoc=false,  % page numbers in ToC flushed right
  eulerchapternumbers=true, % use AMS Euler for chapter font (otherwise Palatino)
  linedheaders=false,       % chaper headers will have line above and beneath
  floatperchapter=true,     % numbering per chapter for all floats (i.e., Figure 1.1)
  minionprospacing=true,
  eulermath=true,  % use awesome Euler fonts for mathematical formulae (only with pdfLaTeX)
  beramono=true,    % toggle a nice monospaced font (w/ bold)
  palatino=true,    % deactivate standard font for loading another one, see the last section at the end of this file for suggestions
  style=arsclassica % classicthesis, arsclassica
}{classicthesis}


% ********************************************************************
% Available options for classicthesis.sty 
% (see ClassicThesis.pdf for more information):
% drafting
% parts nochapters linedheaders
% eulerchapternumbers beramono eulermath pdfspacing minionprospacing
% tocaligned dottedtoc manychapters
% listings floatperchapter subfig
% ********************************************************************

\usepackage{url}
\usepackage{nameref}
\usepackage{epigraph} %used for neat epigraphs at begging of chapters
\usepackage{verse}
\usepackage{datetime} %needed to add the date just below
\usepackage{pdflscape} %used to rotate pages with wide tables
\usepackage{rotating} %used to rotate wide figures
\usepackage{siunitx} %used for nice number formatting
\usepackage{pgf} % needed for arrays among other things
% ****************************************************************************************************
% 2. Personal data and user ad-hoc commands
% ****************************************************************************************************

\renewcommand{\epigraphsize}{\scriptsize}

\def\authors{
  {Student No 1/Student ID 1},
  {Student No 2/Student ID 2},
  {Student No 3/Student ID 3},
  {Student No 4/Student ID 4}}
\def\myTitle{Experimental CS Thesis Title\xspace}
\def\mySubtitle{A Guide for Content \& Style\xspace}
\def\myDegree{Master's Thesis\xspace}
\def\myShortNames{Student 1, 2, 3 \& 4\xspace}
\def\myGroup{My Group (if applicable)}
%\newcommand{\myStudentId}{20051234\xspace}
%\newcommand{\myNameToo}{Put Other Student Name Here\xspace}
%\newcommand{\myStudentIdToo}{20051235\xspace}
\def\myProf{Niels Olof Bouvin\xspace}
\def\myOtherProf{Put name here\xspace}
\def\mySupervisor{Put name here\xspace}
\def\myFaculty{Faculty of Natural Sciences\xspace}
\def\myDepartment{Department of Computer Science\xspace}
\def\myUni{Aarhus University\xspace}
\def\myLocation{Aarhus\xspace}
\def\myTime{\monthname\ \the\year\xspace}
\def\myVersion{\xspace}


% ********************************************************************
% Setup, finetuning, and useful commands
% ********************************************************************
\newcounter{dummy} % necessary for correct hyperlinks (to index, bib, etc.)
\newlength{\abcd} % for ab..z string length calculation
\providecommand{\mLyX}{L\kern-.1667em\lower.25em\hbox{Y}\kern-.125emX\@}
\providecommand{\mBibTeX}{\textsc{Bib}\negthinspace\TeX\@}
% ****************************************************************************************************


% ****************************************************************************************************
% 3. Loading some handy packages
% ****************************************************************************************************
% ******************************************************************** 
% Packages with options that might require adjustments
% ******************************************************************** 
%\PassOptionsToPackage{ngerman,american}{babel}   % change this to your language(s)
% Spanish languages need extra options in order to work with this template
%\PassOptionsToPackage{spanish,es-lcroman}{babel}
\usepackage[american]{babel}                  
\usepackage{microtype}
\usepackage{csquotes}
\usepackage{ragged2e}
% \usepackage[usenames,dvipsnames]{xcolor}
\usepackage{xcolor}
\usepackage{pgf-umlsd}
\usepackage{pgfplots}
\usepackage{pgfplotstable}
% recommended:
\usepackage{booktabs}
\usepackage{array}
\usepackage{colortbl}
\usepackage{amsthm}

% \usepackage{color}
\definecolor{aublue}{HTML}{193b77} % the dark blue colour used by AU

% Modify the biblabel format to use aublue for the source attribution
\renewcommand{\mkcitation}[1]{\xspace\textcolor{aublue}{[#1]}}




%\usepackage[square,numbers,sort&compress]{natbib}
%\bibliographystyle{abbrvnat}

\PassOptionsToPackage{%
   %backend=biber, %instead of bibtex
	backend=bibtex8,bibencoding=ascii,%
	language=auto,%
	style=numeric-comp,%
   %style=authoryear-comp, % Author 1999, 2010
   %bibstyle=authoryear,dashed=false, % dashed: substitute rep. author with ---
   sorting=nyt, % name, year, title
   maxbibnames=10, % default: 3, et al.
   %backref=true,%
   natbib=true % natbib compatibility mode (\citep and \citet still work)
}{biblatex}
\usepackage{biblatex}

\PassOptionsToPackage{fleqn}{amsmath}       % math environments and more by the AMS 
\usepackage{amsmath}
%
% ******************************************************************** 
% General useful packages
% ******************************************************************** 
\PassOptionsToPackage{T1}{fontenc} % T2A for cyrillics
\usepackage{fontenc}   
\usepackage{comment} 
\usepackage{textcomp} % fix warning with missing font shapes
\usepackage{scrhack} % fix warnings when using KOMA with listings package          
\usepackage{xspace} % to get the spacing after macros right  
\newcommand{\eg}{e.g.,\xspace}
\newcommand{\Eg}{E.g.,\xspace}
\newcommand{\etal}{et~al.\xspace}
\newcommand{\etc}{etc.\@\xspace}
\newcommand{\ie}{i.e.,\xspace}
\newcommand{\viz}{viz.\@\xspace}

\usepackage{mparhack} % get marginpar right
\usepackage{fixltx2e} % fixes some LaTeX stuff --> since 2015 in the LaTeX kernel (see below)
%\usepackage[latest]{latexrelease} % will be used once available in more distributions (ISSUE #107)
\PassOptionsToPackage{printonlyused,smaller}{acronym} 
    \usepackage{acronym} % nice macros for handling all acronyms in the thesis
    %\renewcommand{\bflabel}[1]{{#1}\hfill} % fix the list of acronyms --> no longer working
    %\renewcommand*{\acsfont}[1]{\textsc{#1}} 
    \renewcommand*{\aclabelfont}[1]{\acsfont{#1}}
% ****************************************************************************************************


% ****************************************************************************************************
% 4. Setup floats: tables, (sub)figures, and captions
% ****************************************************************************************************
\usepackage{tabularx} % better tables
    \setlength{\extrarowheight}{3pt} % increase table row height
\newcommand{\tableheadline}[1]{\multicolumn{1}{c}{\spacedlowsmallcaps{#1}}}
\newcommand{\myfloatalign}{\centering} % to be used with each float for alignment
\usepackage{caption}
% Thanks to cgnieder and Claus Lahiri
% http://tex.stackexchange.com/questions/69349/spacedlowsmallcaps-in-caption-label
% [REMOVED DUE TO OTHER PROBLEMS, SEE ISSUE #82]    
%\DeclareCaptionLabelFormat{smallcaps}{\bothIfFirst{#1}{~}\MakeTextLowercase{\textsc{#2}}}
%\captionsetup{font=small,labelformat=smallcaps} % format=hang,
\captionsetup{font=small} % format=hang,
\usepackage{subfig}  
% ****************************************************************************************************


% ****************************************************************************************************
% 5. Setup code listings
% ****************************************************************************************************
\usepackage{listings}
%\lstset{emph={trueIndex,root},emphstyle=\color{BlueViolet}}%\underbar} % for special keywords
\lstset{language=[LaTeX]Tex,%C++,
    keywordstyle=\bfseries,%\color{RoyalBlue},%\bfseries,
    basicstyle=\footnotesize\ttfamily,
    identifierstyle=\color{RoyalBlue},
    commentstyle=\color{Green}\itshape,
    stringstyle=\sffamily,
    numbers=left,
    numberstyle=\tiny,
    stepnumber=1,
    numbersep=8pt,
    showstringspaces=false,
    breaklines=true,
    frameround=ffft,
    belowcaptionskip=.75\baselineskip,
    frame=single,
    rulecolor=\color{aublue}
}
% ****************************************************************************************************             


% ****************************************************************************************************
% 6. PDFLaTeX, hyperreferences and citation backreferences
% ****************************************************************************************************
% ********************************************************************
% Using PDFLaTeX
% ********************************************************************
\PassOptionsToPackage{pdftex,hyperfootnotes=false,pdfpagelabels}{hyperref}
    \usepackage{hyperref}  % backref linktocpage pagebackref
\pdfcompresslevel=9
\pdfadjustspacing=1 
\PassOptionsToPackage{pdftex}{graphicx}
    \usepackage{graphicx} 
 

% ********************************************************************
% Hyperreferences
% ********************************************************************
\hypersetup{%
    %draft, % = no hyperlinking at all (useful in b/w printouts)
    colorlinks=true, linktocpage=true, pdfstartpage=3, pdfstartview=FitV,%
    % uncomment the following line if you want to have black links (e.g., for printing)
    %colorlinks=false, linktocpage=false, pdfstartpage=3, pdfstartview=FitV, pdfborder={0 0 0},%
    breaklinks=true, pdfpagemode=UseNone, pageanchor=true, pdfpagemode=UseOutlines,%
    plainpages=false, bookmarksnumbered, bookmarksopen=true, bookmarksopenlevel=1,%
    hypertexnames=true, pdfhighlight=/O,%nesting=true,%frenchlinks,%
    urlcolor=webbrown, linkcolor=RoyalBlue, citecolor=webgreen, %pagecolor=RoyalBlue,%
    %urlcolor=Black, linkcolor=Black, citecolor=Black, %pagecolor=Black,%
    pdftitle={\myTitle},%
    pdfauthor={\textcopyright\ \myShortNames, \myUni, \myFaculty},%
    pdfsubject={},%
    pdfkeywords={},%
    pdfcreator={pdfLaTeX},%
    pdfproducer={LaTeX with hyperref and classicthesis}%
}   

% ********************************************************************
% Setup autoreferences
% ********************************************************************
% There are some issues regarding autorefnames
% http://www.ureader.de/msg/136221647.aspx
% http://www.tex.ac.uk/cgi-bin/texfaq2html?label=latexwords
% you have to redefine the makros for the 
% language you use, e.g., american, ngerman
% (as chosen when loading babel/AtBeginDocument)
% ********************************************************************
\makeatletter
\@ifpackageloaded{babel}%
    {%
       \addto\extrasamerican{%
			\renewcommand*{\figureautorefname}{Figure}%
			\renewcommand*{\tableautorefname}{Table}%
			\renewcommand*{\partautorefname}{Part}%
			\renewcommand*{\chapterautorefname}{Chapter}%
			\renewcommand*{\sectionautorefname}{Section}%
			\renewcommand*{\subsectionautorefname}{Section}%
			\renewcommand*{\subsubsectionautorefname}{Section}%     
                }%
       \addto\extrasngerman{% 
			\renewcommand*{\paragraphautorefname}{Absatz}%
			\renewcommand*{\subparagraphautorefname}{Unterabsatz}%
			\renewcommand*{\footnoteautorefname}{Fu\"snote}%
			\renewcommand*{\FancyVerbLineautorefname}{Zeile}%
			\renewcommand*{\theoremautorefname}{Theorem}%
			\renewcommand*{\appendixautorefname}{Anhang}%
			\renewcommand*{\equationautorefname}{Gleichung}%        
			\renewcommand*{\itemautorefname}{Punkt}%
                }%  
            % Fix to getting autorefs for subfigures right (thanks to Belinda Vogt for changing the definition)
            \providecommand{\subfigureautorefname}{\figureautorefname}%             
    }{\relax}
\makeatother


% ****************************************************************************************************
% 7. Last calls before the bar closes
% ****************************************************************************************************
% ********************************************************************
% Development Stuff
% ********************************************************************
\listfiles
%\PassOptionsToPackage{l2tabu,orthodox,abort}{nag}
%   \usepackage{nag}
%\PassOptionsToPackage{warning, all}{onlyamsmath}
%   \usepackage{onlyamsmath}

% ********************************************************************
% Last, but not least...
% ********************************************************************
\usepackage{classicthesis} 
% ****************************************************************************************************


% ****************************************************************************************************
% 8. Further adjustments (experimental)
% ****************************************************************************************************
% ********************************************************************
% Changing the text area
% ********************************************************************
%\linespread{1.05} % a bit more for Palatino
%\areaset[current]{312pt}{761pt} % 686 (factor 2.2) + 33 head + 42 head \the\footskip
%\setlength{\marginparwidth}{7em}%
%\setlength{\marginparsep}{2em}%

% ********************************************************************
% Using different fonts
% ********************************************************************
%\usepackage[oldstylenums]{kpfonts} % oldstyle notextcomp
%\usepackage[osf]{libertine}
%\usepackage[light,condensed,math]{iwona}
%\renewcommand{\sfdefault}{iwona}
%\usepackage{lmodern} % <-- no osf support :-(
%\usepackage{cfr-lm} % 
%\usepackage[urw-garamond]{mathdesign} <-- no osf support :-(
%\usepackage[default,osfigures]{opensans} % scale=0.95 
%\usepackage[sfdefault]{FiraSans}
% ****************************************************************************************************



%%% Local Variables:
%%% mode: latex
%%% TeX-master: "./ClassicThesis"
%%% End:

% ****************************************************************************************************  
% If you like the classicthesis, then I would appreciate a postcard. 
% My address can be found in the file ClassicThesis.pdf. A collection 
% of the postcards I received so far is available online at 
% http://postcards.miede.de
% ****************************************************************************************************


% ****************************************************************************************************
% 0. Set the encoding of your files. UTF-8 is the only sensible encoding nowadays. If you can't read
% äöüßáéçèê∂åëæƒÏ€ then change the encoding setting in your editor, not the line below. If your editor
% does not support utf8 use another editor!
% ****************************************************************************************************
\PassOptionsToPackage{usenames,dvipsnames}{xcolor}%needed because several packages include xcolor
\PassOptionsToPackage{utf8}{inputenc}
\usepackage{inputenc}
        
% ****************************************************************************************************
% 1. Configure classicthesis for your needs here, e.g., remove "drafting" below 
% in order to deactivate the time-stamp on the pages
% ****************************************************************************************************

% \PassOptionsToPackage{eulerchapternumbers,listings,%drafting,%
%					 pdfspacing,%floatperchapter,%linedheaders,%
%					 subfig,beramono,eulermath,parts}{arsclassica}

\PassOptionsToPackage{
  drafting=false,    % print version information on the bottom of the pages
  tocaligned=false, % the left column of the toc will be aligned (no indentation)
  dottedtoc=false,  % page numbers in ToC flushed right
  eulerchapternumbers=true, % use AMS Euler for chapter font (otherwise Palatino)
  linedheaders=false,       % chaper headers will have line above and beneath
  floatperchapter=true,     % numbering per chapter for all floats (i.e., Figure 1.1)
  minionprospacing=true,
  eulermath=true,  % use awesome Euler fonts for mathematical formulae (only with pdfLaTeX)
  beramono=true,    % toggle a nice monospaced font (w/ bold)
  palatino=true,    % deactivate standard font for loading another one, see the last section at the end of this file for suggestions
  style=arsclassica % classicthesis, arsclassica
}{classicthesis}


% ********************************************************************
% Available options for classicthesis.sty 
% (see ClassicThesis.pdf for more information):
% drafting
% parts nochapters linedheaders
% eulerchapternumbers beramono eulermath pdfspacing minionprospacing
% tocaligned dottedtoc manychapters
% listings floatperchapter subfig
% ********************************************************************

\usepackage{url}
\usepackage{nameref}
\usepackage{epigraph} %used for neat epigraphs at begging of chapters
\usepackage{verse}
\usepackage{datetime} %needed to add the date just below
\usepackage{pdflscape} %used to rotate pages with wide tables
\usepackage{rotating} %used to rotate wide figures
\usepackage{siunitx} %used for nice number formatting
\usepackage{pgf} % needed for arrays among other things
\usepackage{amsthm}
\usepackage[noend, noline, ruled, linesnumbered]{algorithm2e}
\usepackage{float}


\newtheorem*{problem*}{Problem}
\newtheorem{problem}{Problem}
\providecommand*{\problemautorefname}{Problem}

\theoremstyle{definition}
\newtheorem{definition}{Definition}
\providecommand*{\definitionautorefname}{Definition}


\SetAlgorithmName{Algorithm}{Algorithm}



% ****************************************************************************************************
% 2. Personal data and user ad-hoc commands
% ****************************************************************************************************

\renewcommand{\epigraphsize}{\scriptsize}

\def\authors{
  {Student No 1/Student ID 1},
  {Student No 2/Student ID 2},
  {Student No 3/Student ID 3},
  {Student No 4/Student ID 4}}
\def\myTitle{Experimental CS Thesis Title\xspace}
\def\mySubtitle{A Guide for Content \& Style\xspace}
\def\myDegree{Master's Thesis\xspace}
\def\myShortNames{Student 1, 2, 3 \& 4\xspace}
%\def\myGroup{My Group (if applicable)}
%\newcommand{\myStudentId}{20051234\xspace}
%\newcommand{\myNameToo}{Put Other Student Name Here\xspace}
%\newcommand{\myStudentIdToo}{20051235\xspace}
\def\myProf{Ioannis Caragiannis\xspace}
\def\myOtherProf{Put name here\xspace}
\def\mySupervisor{Put name here\xspace}
\def\myFaculty{Faculty of Natural Sciences\xspace}
\def\myDepartment{Department of Computer Science\xspace}
\def\myUni{Aarhus University\xspace}
\def\myLocation{Aarhus\xspace}
\def\myTime{\monthname\ \the\year\xspace}
\def\myVersion{\xspace}


% ********************************************************************
% Setup, finetuning, and useful commands
% ********************************************************************
\newcounter{dummy} % necessary for correct hyperlinks (to index, bib, etc.)
\newlength{\abcd} % for ab..z string length calculation
\providecommand{\mLyX}{L\kern-.1667em\lower.25em\hbox{Y}\kern-.125emX\@}
\providecommand{\mBibTeX}{\textsc{Bib}\negthinspace\TeX\@}
% ****************************************************************************************************


% ****************************************************************************************************
% 3. Loading some handy packages
% ****************************************************************************************************
% ******************************************************************** 
% Packages with options that might require adjustments
% ******************************************************************** 
%\PassOptionsToPackage{ngerman,american}{babel}   % change this to your language(s)
% Spanish languages need extra options in order to work with this template
%\PassOptionsToPackage{spanish,es-lcroman}{babel}
\usepackage[american]{babel}                  
\usepackage{microtype}
\usepackage{csquotes}
\usepackage{ragged2e}
% \usepackage[usenames,dvipsnames]{xcolor}
\usepackage{xcolor}
\usepackage{pgf-umlsd}
\usepackage{pgfplots}
\usepackage{pgfplotstable}
% recommended:
\usepackage{booktabs}
\usepackage{array}
\usepackage{colortbl}
\usepackage{amsthm}

% \usepackage{color}
\definecolor{aublue}{HTML}{193b77} % the dark blue colour used by AU

% Modify the biblabel format to use aublue for the source attribution
\renewcommand{\mkcitation}[1]{\xspace\textcolor{aublue}{[#1]}}




%\usepackage[square,numbers,sort&compress]{natbib}
%\bibliographystyle{abbrvnat}

\PassOptionsToPackage{%
   %backend=biber, %instead of bibtex
	backend=bibtex8,bibencoding=ascii,%
	language=auto,%
	style=numeric-comp,%
   %style=authoryear-comp, % Author 1999, 2010
   %bibstyle=authoryear,dashed=false, % dashed: substitute rep. author with ---
   sorting=nyt, % name, year, title
   maxbibnames=10, % default: 3, et al.
   %backref=true,%
   natbib=true % natbib compatibility mode (\citep and \citet still work)
}{biblatex}
\usepackage{biblatex}

\PassOptionsToPackage{fleqn}{amsmath}       % math environments and more by the AMS 
\usepackage{amsmath}
%
% ******************************************************************** 
% General useful packages
% ******************************************************************** 
\PassOptionsToPackage{T1}{fontenc} % T2A for cyrillics
\usepackage{fontenc}   
\usepackage{comment} 
\usepackage{textcomp} % fix warning with missing font shapes
\usepackage{scrhack} % fix warnings when using KOMA with listings package          
\usepackage{xspace} % to get the spacing after macros right  
\newcommand{\eg}{e.g.,\xspace}
\newcommand{\Eg}{E.g.,\xspace}
\newcommand{\etal}{et~al.\xspace}
\newcommand{\etc}{etc.\@\xspace}
\newcommand{\ie}{i.e.,\xspace}
\newcommand{\viz}{viz.\@\xspace}

\usepackage{mparhack} % get marginpar right
\usepackage{fixltx2e} % fixes some LaTeX stuff --> since 2015 in the LaTeX kernel (see below)
%\usepackage[latest]{latexrelease} % will be used once available in more distributions (ISSUE #107)
\PassOptionsToPackage{printonlyused,smaller}{acronym} 
    \usepackage{acronym} % nice macros for handling all acronyms in the thesis
    %\renewcommand{\bflabel}[1]{{#1}\hfill} % fix the list of acronyms --> no longer working
    %\renewcommand*{\acsfont}[1]{\textsc{#1}} 
    \renewcommand*{\aclabelfont}[1]{\acsfont{#1}}
% ****************************************************************************************************


% ****************************************************************************************************
% 4. Setup floats: tables, (sub)figures, and captions
% ****************************************************************************************************
\usepackage{tabularx} % better tables
    \setlength{\extrarowheight}{3pt} % increase table row height
\newcommand{\tableheadline}[1]{\multicolumn{1}{c}{\spacedlowsmallcaps{#1}}}
\newcommand{\myfloatalign}{\centering} % to be used with each float for alignment
\usepackage{caption}
% Thanks to cgnieder and Claus Lahiri
% http://tex.stackexchange.com/questions/69349/spacedlowsmallcaps-in-caption-label
% [REMOVED DUE TO OTHER PROBLEMS, SEE ISSUE #82]    
%\DeclareCaptionLabelFormat{smallcaps}{\bothIfFirst{#1}{~}\MakeTextLowercase{\textsc{#2}}}
%\captionsetup{font=small,labelformat=smallcaps} % format=hang,
\captionsetup{font=small} % format=hang,
\usepackage{subfig}  
% ****************************************************************************************************


% ****************************************************************************************************
% 5. Setup code listings
% ****************************************************************************************************
\usepackage{listings}
%\lstset{emph={trueIndex,root},emphstyle=\color{BlueViolet}}%\underbar} % for special keywords
\lstset{language=[LaTeX]Tex,%C++,
    keywordstyle=\bfseries,%\color{RoyalBlue},%\bfseries,
    basicstyle=\footnotesize\ttfamily,
    identifierstyle=\color{RoyalBlue},
    commentstyle=\color{Green}\itshape,
    stringstyle=\sffamily,
    numbers=left,
    numberstyle=\tiny,
    stepnumber=1,
    numbersep=8pt,
    showstringspaces=false,
    breaklines=true,
    frameround=ffft,
    belowcaptionskip=.75\baselineskip,
    frame=single,
    rulecolor=\color{aublue}
}
% ****************************************************************************************************             


% ****************************************************************************************************
% 6. PDFLaTeX, hyperreferences and citation backreferences
% ****************************************************************************************************
% ********************************************************************
% Using PDFLaTeX
% ********************************************************************
\PassOptionsToPackage{pdftex,hyperfootnotes=false,pdfpagelabels}{hyperref}
    \usepackage{hyperref}  % backref linktocpage pagebackref
\pdfcompresslevel=9
\pdfadjustspacing=1 
\PassOptionsToPackage{pdftex}{graphicx}
    \usepackage{graphicx} 
 

% ********************************************************************
% Hyperreferences
% ********************************************************************
\hypersetup{%
    %draft, % = no hyperlinking at all (useful in b/w printouts)
    colorlinks=true, linktocpage=true, pdfstartpage=3, pdfstartview=FitV,%
    % uncomment the following line if you want to have black links (e.g., for printing)
    %colorlinks=false, linktocpage=false, pdfstartpage=3, pdfstartview=FitV, pdfborder={0 0 0},%
    breaklinks=true, pdfpagemode=UseNone, pageanchor=true, pdfpagemode=UseOutlines,%
    plainpages=false, bookmarksnumbered, bookmarksopen=true, bookmarksopenlevel=1,%
    hypertexnames=true, pdfhighlight=/O,%nesting=true,%frenchlinks,%
    urlcolor=webbrown, linkcolor=RoyalBlue, citecolor=webgreen, %pagecolor=RoyalBlue,%
    %urlcolor=Black, linkcolor=Black, citecolor=Black, %pagecolor=Black,%
    pdftitle={\myTitle},%
    pdfauthor={\textcopyright\ \myShortNames, \myUni, \myFaculty},%
    pdfsubject={},%
    pdfkeywords={},%
    pdfcreator={pdfLaTeX},%
    pdfproducer={LaTeX with hyperref and classicthesis}%
}   

% ********************************************************************
% Setup autoreferences
% ********************************************************************
% There are some issues regarding autorefnames
% http://www.ureader.de/msg/136221647.aspx
% http://www.tex.ac.uk/cgi-bin/texfaq2html?label=latexwords
% you have to redefine the makros for the 
% language you use, e.g., american, ngerman
% (as chosen when loading babel/AtBeginDocument)
% ********************************************************************
\makeatletter
\@ifpackageloaded{babel}%
    {%
       \addto\extrasamerican{%
			\renewcommand*{\figureautorefname}{Figure}%
			\renewcommand*{\tableautorefname}{Table}%
			\renewcommand*{\partautorefname}{Part}%
			\renewcommand*{\chapterautorefname}{Chapter}%
			\renewcommand*{\sectionautorefname}{Section}%
			\renewcommand*{\subsectionautorefname}{Section}%
			\renewcommand*{\subsubsectionautorefname}{Section}%     
                }%
       \addto\extrasngerman{% 
			\renewcommand*{\paragraphautorefname}{Absatz}%
			\renewcommand*{\subparagraphautorefname}{Unterabsatz}%
			\renewcommand*{\footnoteautorefname}{Fu\"snote}%
			\renewcommand*{\FancyVerbLineautorefname}{Zeile}%
			\renewcommand*{\theoremautorefname}{Theorem}%
			\renewcommand*{\appendixautorefname}{Anhang}%
			\renewcommand*{\equationautorefname}{Gleichung}%        
			\renewcommand*{\itemautorefname}{Punkt}%
                }%  
            % Fix to getting autorefs for subfigures right (thanks to Belinda Vogt for changing the definition)
            \providecommand{\subfigureautorefname}{\figureautorefname}%             
    }{\relax}
\makeatother


% ****************************************************************************************************
% 7. Last calls before the bar closes
% ****************************************************************************************************
% ********************************************************************
% Development Stuff
% ********************************************************************
\listfiles
%\PassOptionsToPackage{l2tabu,orthodox,abort}{nag}
%   \usepackage{nag}
%\PassOptionsToPackage{warning, all}{onlyamsmath}
%   \usepackage{onlyamsmath}

% ********************************************************************
% Last, but not least...
% ********************************************************************
\usepackage{classicthesis} 
% ****************************************************************************************************


% ****************************************************************************************************
% 8. Further adjustments (experimental)
% ****************************************************************************************************
% ********************************************************************
% Changing the text area
% ********************************************************************
%\linespread{1.05} % a bit more for Palatino
%\areaset[current]{312pt}{761pt} % 686 (factor 2.2) + 33 head + 42 head \the\footskip
%\setlength{\marginparwidth}{7em}%
%\setlength{\marginparsep}{2em}%

% ********************************************************************
% Using different fonts
% ********************************************************************
%\usepackage[oldstylenums]{kpfonts} % oldstyle notextcomp
%\usepackage[osf]{libertine}
%\usepackage[light,condensed,math]{iwona}
%\renewcommand{\sfdefault}{iwona}
%\usepackage{lmodern} % <-- no osf support :-(
%\usepackage{cfr-lm} % 
%\usepackage[urw-garamond]{mathdesign} <-- no osf support :-(
%\usepackage[default,osfigures]{opensans} % scale=0.95 
%\usepackage[sfdefault]{FiraSans}
% ****************************************************************************************************



%%% Local Variables:
%%% mode: latex
%%% TeX-master: "./ClassicThesis"
%%% End:
