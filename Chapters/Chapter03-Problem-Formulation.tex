\chapter{Problem Formulation}
\label{cha:problem_formulation}

This chapter introduces the formal model that underpins this thesis. We consider a generalization of the standard kidney exchange framework, where each patient may be associated with more than one donor. We begin by reviewing the conventional patient-donor pair model and then extend it to accommodate the setting with multiple donors per patient. We subsequently define a series of algorithmic problems that emerge in this multi-donor context. These problems depart from classical kidney exchange formulations and can be interpreted from multiple perspectives, including structured graph matching, weighted set packing, and integer programming. This chapter establishes the theoretical foundation for the complexity analysis, algorithm design, and experimental evaluation presented in the subsequent chapters.

\section{Overview of the Model}

We begin by reviewing two conventional models for representing patient-donor pairs, which are commonly used in kidney exchange literature.

In the first model, each node represents a patient-donor pair, and each directed edge indicates that the patient in the source node is compatible with the donor in the target node. This results in an unweighted directed graph $G = (V, E)$. Depending on the specific problem setting — such as restrictions on the length of allowed exchange cycles — this formulation leads to different variants of graph matching problems.

The second model takes a slightly different approach by separating patients and donors into distinct nodes. For each patient-donor pair, an edge is added from the patient node to the donor node. Additionally, for each compatible donor-patient relationship, a directed edge is added from the donor to the compatible patient. The resulting graph is a directed bipartite graph. Notably, even though the nodes initially represent different roles (patients or donors), the structure of the graph itself encodes this information. When computing matchings, the graph $G = (V, E)$ contains all necessary information; we do not need to explicitly distinguish whether a node is a patient or a donor.

% add a simple image example here
% maybe we do not need The second model here

What happens if we introduce second donors? We consider two distinct scenarios.

In the first scenario, each patient may be associated with multiple alternative donors, but since a patient only needs one kidney, only one donor's kidney will actually be used in the \ac{KE}. This setting is nearly identical to the standard patient-donor pair \ac{KE} problem: the objective is to find disjoint exchange cycles in a directed compatibility graph.

However, a more general scenario arises if we allow a patient to receive a kidney while both of their associated donors are permitted to donate. In this case, the first donor may participate in an exchange cycle as usual, while the second donor initiates a donation chain, effectively increasing the total number of transplants.

% insert a figure here to illustrate a cycle followed by a chain

In such settings, we observe that allowing chains to follow cycles can only improve social welfare—it never reduces the total number of transplants. Moreover, this problem deviates from the classical graph matching framework: it is no longer a pure matching problem.

Finally, we note that it is now possible for one donor to participate in a cycle while the second donor from the same patient initiates or joins a chain. This creates a structural asymmetry in the problem. A general graph representation is no longer sufficient, as all nodes and edges are treated identically. Instead, we need to explicitly distinguish between patients and donors in the model to accurately capture the structure and constraints of the problem.

\section{Formal Definitions}
We model the problem as a directed graph $G(V, E)$. The set of vertices $V = P \cup D$ consists of two subsets of nodes, first representing patients $P$, second representing donors $D$. The set of edges consists of two subsets $E = E_{proxy} \cup E_{compatible}$:
\begin{itemize}
    \item The subset of edges $E_{proxy} \subseteq E$ contains edges $d \leftarrow p$ for some $d \in D$ and $p \in P$ indicating that donor $d \in D$ is a proxy donor of patient $p \in P$.
    \item the subset of edges $E_{compatible} \subseteq E$ contains edges $p \leftarrow d$ indicating that a patient $p \in P$ is compatible with donor $d \in D$.
\end{itemize}
In this model, a donor who is a proxy for a patient cannot simultaneously be compatible with that same patient. Each patient may be compatible with multiple donors but can have at most two proxy donors. Each donor serves as a proxy for exactly one patient.\\
The objective is to find subset of edges $E_{opt} \subseteq E$ such that:
\begin{itemize}
    \item $\forall p \in P, \left|\{ d : p \leftarrow d \in E_{opt} \}\right| \le 1$, that is, each patient can receive only one compatible kidney.
    \item $\forall d \in D$ there exists some $ p \leftarrow d \in E_{opt}$ only if $p' \leftarrow d' \in E_{opt}$ and $d \leftarrow p' \in E_{proxy}$, in other words, a donor can be matched with a compatible patient only if their proxy patient also receives a compatible kidney.
    \item the number of compatible edges $|E_{opt}|$ is maximized.
\end{itemize}

\section{Generalized Problems}

The model introduced above captures the essential structure of multi-donor kidney exchange and donation. We now build upon this model to define specific algorithmic problems under various settings. The remainder of this paper will focus on analyzing and solving these problems.

As a starting point, consider the most general setting, where no restrictions are imposed on the length of exchange cycles or donation chains. Given a directed graph $G = (V = P \cup D, E)$, where $P$ denotes patients and $D$ denotes donors, all subsets of edges that satisfy the feasibility constraints described earlier are considered valid solutions. We refer to this formulation as the \textit{Generalized Multi-Donor Exchange Problem} (placeholder name).

We then consider a more constrained version, where limits are placed on the maximum allowed lengths of cycles and chains. Specifically, in the \textit{$n$-Cycle $m$-Chain Problem}, only subsets in which all cycles have length at most $n$ and all chains have length at most $m$ are considered valid.



\section{Alternative Interpretations}





%%% Local Variables:
%%% mode: latex
%%% TeX-master: "../ClassicThesis"
%%% ispell-dictionary: "british" ***
%%% fill-column: 76 ***
%%% End:
