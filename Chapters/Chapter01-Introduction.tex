\chapter{Introduction}
\label{cha:introduction}
\epigraph{``Curiouser and curiouser!'' cried Alice.}{\textit{Alice’s Adventures in Wonderland}\\\textsc{Lewis Carroll}}

The purpose of \autoref{cha:introduction} is make a short (2--6 pages)
argument that should cover
\begin{aenumerate}
\item What this thesis is about?
\item Why it is interesting or important?
\item What are the central hypotheses that will be investigated?
\item How will the work be done?
\end{aenumerate}

This is the place where the reader (who will be a computer scientist, but
might not be a domain expert) should be convinced that not only is the topic
interesting and important, the authors have also identified central
questions/hypotheses pertaining to the topic, and have a clear plan and
methodology to address it.

\section{What Makes a Good Hypothesis?}
\label{sec:what-makes-good}

A hypothesis must be a ``good question'' in the sense that the question is
an interesting one and that the answer should not be obvious to a computer
scientist (or CS/IT student).  If you know the answers in advance, you
really should be asking different questions.

For the purposes of a report or thesis, it is wise to concentrate on
research questions and hypotheses that are decidable or quantifiable. \Eg it
is better to state that ``method A is better than method B under
circumstances C'' or ``combining method A with architecture B improves on
standard approach C'' than ``we can build a system that does X''.  This is
why it is always a good idea to include baselines in your work, \ie
established methods or architectural choices that can used for
comparison. If you do not have baselines yourself, you should at least be
ready and able to compare your results with the published results of others.

If your thesis work is exploring so-called a ``wicked problem\footnote{\url{https://en.wikipedia.org/wiki/Wicked_problem} accessed 2024/11/22}'', the validation
of your work will rely on other criteria than quantifiable measurements and
rejections of hypotheses.  Research through design is a young field and
quality criteria are currently debated and developed. You may work with
\citeauthor{Zimmerman2007:POTSCOHFICS2007}~\cite{Zimmerman2007:POTSCOHFICS2007}, draw upon
\citeauthor{Gaver2012:POTSCOHFICS2012}~\cite{Gaver2012:POTSCOHFICS2012} and work with artefacts as theory
nexus, or as annotated portfolios. Alternatively, you can approach HCI
research as problem solving, as suggested by~\citeauthor{Oulasvirta2016:POT2CCOHFICS2016}:

\begin{displayquote}[Definition 1:~\cite{Oulasvirta2016:POT2CCOHFICS2016}]
  A research problem in HCI is a stated lack of understanding about some
  phenomenon in human use of computing, or stated inability to construct
  interactive technology to address that phenomenon for desired
  ends.
\end{displayquote}


The hypotheses should address central aspects of the work, so that \emph{if}
these hypotheses are met, the overall work gains in credibility, or
alternatively (and just as valid), if a hypothesis \emph{cannot} be
confirmed, it illustrates why and how the assumptions behind the work were
flawed, and, hopefully, how they can be improved.



\section{Writing a Thesis Worth Reading}
\label{sec:writ-thes-read}

The purpose of the thesis is to be read as a whole in one sitting, and with
this in mind it should be written, even if, in reality, it is authored over
a period of months.  The reader does not naturally understand the flow and
process of the work involved (this understanding, hopefully, belongs to the
authors), and must therefore be guided by the authors through the work.  In
order to accomplish this, the readers should at all times have a ready
answer in their mind to the following questions:

\begin{itemize}
\item Why am I reading this?
\item What comes next?
\item How does this build upon what I just read?
\end{itemize}

So, why is something there? What is its purpose? How will it used later?
Vice versa, later in the text, refer back to things established earlier
(this also supports readers that do not necessarily read linearly). While a
text grows piecemeal, it is most often read as a whole, and should appear as
such, lest the reader loses interest.

It is to that end a good idea to finish the introduction with a description
of how the hypotheses are to be investigated, and how this is reflected in
the structure of the thesis.



\section{(MANDATORY) Generative AI Declaration}
\label{sec:gener-ai-decl}

You should here state \emph{clearly}, how Generative AI has been used in your thesis work, whether it has been as a programming aid, tool to clean-up your text, or tool to generate text or figures. You should throughout the thesis adhere to the advice given below, when it comes to references and attributions.

If you have \emph{not} at \emph{any} point during your work used Generative AI, you should state so here---a declaration is necessary and mandatory.


\subsection{Using Generative AI in Your Thesis}
\label{sec:using-generative-ai}

It is now permitted to use GenAI at AU\footnote{\url{https://studerende.au.dk/en/gai}}, but you must treat this kind of tool with caution as you would any outside help. \emph{You} are, after all, the one writing the thesis, \emph{not} some machine!

\paragraph{Good uses for GenAI} Generative AI, \eg ChatGPT, can be excellent tools for learning about new topics or summarising some area of interest. As such, it can be a powerful supplement to ordinary search engine use and literature perusal. You should however always keep in mind that GenAI is prone to hallucinations, and thus verify any claims made. A \ac{LLM} may seem confident in its output, but it is just an algorithm generating texts probabilistic based on a training corpus, and thus produces texts  \emph{likely} rather than \emph{true}.

Generative AI can be very useful, when it comes to correcting grammar, or cleaning up text. Ideally, you should use GenAI as a tutor providing critique---using its suggestions to improve your own writing, so that you, over time, do not require its assistance. An effective way of doing this is to instruct the \ac{LLM} to behave in a specific way, e.g., preface the text you wish improved with something like this:

\begin{quotation}
  ``I wish you to critique the following Master's Thesis abstract as a seasoned Computer Science professor and thesis supervisor would. You are not to offer a corrected version of the text, but you are to point out strengths and weaknesses, so that the recipient can correct the text themselves.''
\end{quotation}

Voilà: a thesis supervisor in a box!

If you use generative AI to alter and improve your text, you must state so in the introduction.



Incidentally, ChatGPT can also be a great help, when it comes to figuring out how to do something in \LaTeX---here, the machine's ability to ingest vast amounts of documentation, and answer questions based on it becomes \emph{very} useful (as long as you understand that it can still occasionally hallucinate, inventing fictitious commands). 

\paragraph{Questionable uses for GenAI} More problematic is using such tools to generate actual thesis content from prompts. If you do that, you \emph{must} indicate that this is machine generated text and provide the prompt used. This is just like providing a reference to a quotation from a scientific publication: including material without attribution is \emph{plagiarism}, \ie exam fraud.


\begin{displayquote}[ChatGPT V4, 2024-05-08: ``Write a short, succinct paragraph on why it is better to write your own text rather than relying on machine generated ditto in the style of a scientific publication'']\small
  While machine-generated text can efficiently produce content, relying solely on such output may compromise the integrity and authenticity of the work. Crafting original text allows for the infusion of personal insights, nuances, and creativity, tailored to the specific context and audience. This process fosters deeper engagement and understanding, facilitating the conveyance of complex ideas with clarity and precision. Moreover, writing one's own text encourages critical thinking, problem-solving, and mastery of the subject matter, nurturing intellectual growth and fostering a sense of ownership and pride in the work. Therefore, while machine-generated text may serve as a useful tool for generating ideas or assisting in drafting, the act of writing one's own text remains invaluable for cultivating expertise, expression, and scholarly contribution.
\end{displayquote}


%%% Local Variables:
%%% mode: latex
%%% TeX-master: "../ClassicThesis"
%%% ispell-dictionary: "british" ***
%%% fill-column: 76 ***
%%% End:
