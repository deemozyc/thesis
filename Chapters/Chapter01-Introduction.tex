\chapter{Introduction}
\label{cha:introduction}


\section{Background and Motivation}

Kidney transplantation is the most effective treatment for patients with end-stage renal disease. However, due to the severe shortage of deceased-donor organs, living kidney donation has become an increasingly important option. In many cases, a willing donor is biologically incompatible with their intended recipient. To overcome this, kidney-paired donation (KPD) programs have been established, allowing incompatible donor-patient pairs to exchange kidneys with other pairs in a mutually beneficial way.

One well-known model assumes that each patient is associated with a single incompatible donor. A patient may receive a compatible kidney through a pair-wise exchange with another patient-donor pair. This setting can be naturally formulated as a graph problem: each patient-donor pair is represented as a vertex, and an undirected edge is added between two vertices if the donors are mutually compatible with the other’s recipient. The optimal allocation in this model corresponds to a maximum cardinality matching in the resulting undirected graph. A similar formulation is presented in \cite{roth2004kidney}.

% may be intro some one-donor model (like involve hospital)

Most existing KPD mechanisms assume that each patient has exactly one associated donor. Under this assumption, the compatibility relationships between pairs can be modelled as a graph, where two-way exchanges correspond to edges, and the optimal matching can be computed using well-known techniques. This simplification has enabled large-scale algorithmic solutions, but it also introduces limitations.

In practice, some patients may have access to more than one incompatible but willing donor—for example, both a spouse and a sibling. Existing models typically ignore the presence of multiple donors or restrict the system to use only one donor per pair. However, allowing more than one donor to participate introduces new possibilities for improving overall efficiency. At the same time, it raises new algorithmic and strategic questions. How should we model compatibility and exchange opportunities in this extended setting? How complex is it to find optimal allocations? Can we design mechanisms that are efficient and incentive-compatible?

% maybe intro the KDE club paper

This thesis explores these questions by proposing a formal model for KPD with multiple donors per patient, analysing its computational complexity, designing approximation algorithms, and studying strategic behavior under different mechanism designs.

\section{Contributions}

Our main contributions focus on a generalisation of kidney exchange that allows each patient to be associated with more than one donor.

We begin by formalising this multi-donor kidney exchange setting into a series of well-defined algorithmic problems. We show that all of these problems are NP-complete and further establish strong inapproximability results.

For one of the problems, we provide a constant-factor approximation algorithm and prove its approximation guarantee.

Importantly, the algorithmic problems we define are not limited to kidney exchange. They also correspond to graph-like problems and structured weighted set packing problems, which may be applicable in other real-world domains.

Next, from a game-theoretic perspective, we investigate the notion of truthfulness. We begin by providing a formal definition of truthfulness in the context of multi-donor kidney exchange. We then prove that no deterministic algorithm can guarantee truthfulness in this setting. However, we show that by introducing randomisation, certain deterministic algorithms can be transformed into truthful randomised mechanisms.

Finally, we evaluate our approach on a synthetic dataset. Our experiments demonstrate that, in practice, optimal solutions for thousands of patients can be computed using integer programming. Moreover, allowing a second donor results in approximately a $10\%$ increase in the number of patients who receive transplants.


\section{Thesis Organization}



%%% Local Variables:
%%% mode: latex
%%% TeX-master: "../ClassicThesis"
%%% ispell-dictionary: "british" ***
%%% fill-column: 76 ***
%%% End:
