\chapter{Introduction}
\label{cha:introduction}


\section{Background and Motivation}
\label{sec:background_and_motivation}

Kidney transplantation is the most effective treatment for patients with end-stage renal disease. However, due to the severe shortage of deceased-donor organs, living kidney donation has become an increasingly important option. In many cases, a willing donor is biologically incompatible with their intended recipient. To overcome this, several types of kidney exchange (KE) donation programs have been studied and established in practice, allowing incompatible donor-patient pairs to exchange kidneys with other pairs in a mutually beneficial way.

% kidney exchange as top trading cycles
One of the first attempts to model kidney exchange programs was based on the well-studied Top Trading Cycles model, which was later extended in \cite{roth2004kidney} to include Top Trading Cycles and Chains. In these models, each patient-donor pair reports a preference ranking over all donors in the system. Each patient-donor pair is represented as a node in a directed graph, with an edge pointing to their most preferred available donor. The algorithm repeatedly identifies and removes cycles (groups of patient-donors who can exchange kidneys among themselves), updating preferences after each removal. This process continues until no cycles remain, ensuring that every exchange benefits all participants involved.

This class of models exemplifies a truthful mechanism, wherein each patient ultimately receives the best possible outcome according to their preferences. Despite this theoretical appeal, such models are not used in practice due to several limitations. First, each patient must report a preference ranking over all donors. In practice, KE programs involve thousands of donors, making it infeasible for patients to construct such detailed preference lists. Moreover, most patients are indifferent among compatible kidneys, as long as the match is viable. Second, the model permits cycles of unrestricted length. However, in practice, all transplants within a cycle must be performed simultaneously to avoid the risk of a donor backing out after their paired patient has already received a kidney. Consequently, KE programs typically restrict cycle lengths to two or three to ensure logistical feasibility and minimize the risk of failure. Lastly, the model focuses on truthfulness and not on optimizing the number of transplants. In practice, however, maximizing the number of transplants is often the primary objective of kidney exchange programs, as this directly translates to more lives saved.

These issues were addressed in \cite{abraham2007clearing}, where the KE program is modeled as a graph in which \textit{patient-donor pairs} are represented as vertices, and directed edges indicate compatibility between them. The goal is to find an exchange that maximizes the number of patients who receive a transplant, while restricting the length of cycles. When cycles are limited to length two, \cite{abraham2007clearing} shows at the beginning of Section 4 that the problem reduces to a maximum weight matching problem on a bipartite graph, which can be solved in polynomial time. Furthermore, when the restriction on the length of cycles is larger than 2, Theorem 1 in \cite{abraham2007clearing} proves that the associated decision problem becomes NP-complete.

In practice, some patients may have access to more than one incompatible but willing donor — for example, both a spouse and a sibling. Existing models can typically be extended to allow the use of \textit{alternative donors}, with only one is ultimately selected in the solution. However, not many models are designed for many-for-one exchange, where \textit{multiple donors} are willing to donate their kidneys simultaneously in exchange for a compatible kidney for their associated patient. Allowing this type of exchange introduces new possibilities for improving overall efficiency. At the same time, it raises new algorithmic and strategic questions. How should we model compatibility and exchange opportunities in this extended setting? How complex is it to find optimal allocations? Can we design mechanisms that are efficient and incentive-compatible?

One closely related idea permits the creation of \textit{clubs}, groups of multiple patients and donors in which a donor is willing to donate if any patient from their club receives a transplant. A well-defined model building on and expanding this idea was introduced in \cite{farina2017operation}, where, Theorem 1 shows that the decision problem associated with the uncapped generalized kidney exchange clubs problem is NP-complete.

However, while this model provides an expressive foundation, it does not fully address the challenges introduced by allowing multiple donors per patient under stricter feasibility and incentive constraints. This thesis builds on these ideas by proposing a refined formal model for kidney exchange with multiple donors per patient, analyzing its computational complexity, designing approximation algorithms, and studying strategic behavior under different mechanism designs.

\section{Contributions and Thesis Organization}

This thesis explores a generalization of kidney exchange in which each patient may be associated with more than one donor.

\autoref{cha:problem_formulation} is about formalizing this multi-donor kidney exchange setting as a series of well-defined algorithmic problems. Notably, these problems are not specific to kidney exchange and may also arise in other real-world domains involving structured matching or constrained resource allocation.

From a game-theoretic perspective, we further investigate the notion of truthfulness at \autoref{cha:incentives_and_truthfulness}.  We formally define truthfulness in the context of multi-donor kidney exchange and prove that no deterministic algorithm can simultaneously achieve truthfulness and social welfare optimality. However, we demonstrate that by introducing randomization, certain deterministic algorithms can be transformed into truthful randomized mechanisms.

% TODO: add ref later
We then show that a series of well-defined algorithmic problems are NP-complete at \autoref{cha:computational_complexity_analysis} and further establish their approximability limits at \autoref{cha:approximation}. For one of the problems, we design a constant-factor approximation algorithm and formally prove its approximation guarantee.


Finally, we evaluate our approach on synthetic datasets at \autoref{cha:Experiments}. Our experiments show that, in practice, optimal solutions for instances involving thousands of patients can be computed efficiently using integer programming. Furthermore, allowing a second donor per patient leads to approximately a $5\%$ increase in the number of patients who receive transplants.



%%% Local Variables:
%%% mode: latex
%%% TeX-master: "../ClassicThesis"
%%% ispell-dictionary: "british" ***
%%% fill-column: 76 ***
%%% End:
