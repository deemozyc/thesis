\chapter{Conclusion}
\label{cha:conclusion}



\section{Summary}

This thesis introduced and analyzed a generalized model for kidney exchange in which each patient may be associated with up to two proxy donors. We formalized this multi-donor setting into well-defined problems, capturing the structural and strategic complexities arising from allowing multiple donors per patient.

Our main contributions include a detailed computational complexity analysis, proving NP-completeness for various problem variants via reductions from classic hard problems. We also provided constant-factor approximation algorithms and established inapproximability bounds for the most relevant cases. Furthermore, we explored incentive issues and showed how randomized mechanisms can restore truthfulness. Finally, we demonstrated the practical feasibility of solving realistic instances through experiments, showing that optimal solutions can be efficiently computed and that introducing second donors significantly improves match rates.

\section{Limitations and Outlook}

While our work provides a theoretical foundation for kidney exchange with multiple donors, several limitations remain. In real-world kidney exchange programs, many additional factors must be considered. Patients are often prioritized based on factors such as waiting time or medical urgency, and compatibility edges are typically weighted to reflect the probability of a successful transplant rather than being binary. Moreover, patients often participate through hospitals or organizations, rather than individually. Finally, our model assumes a static, one-shot matching process, whereas real kidney exchange programs are dynamic and continuously evolving over time.

Therefore, future work could extend our model to incorporate the practical complexities outlined above. Additionally, in \autoref{cha:approximation}, our inapproximability results are primarily derived via reductions from problems in \autoref{cha:computational_complexity_analysis}. Further insights may be obtained by exploring alternative techniques beyond classical reductions.



%%% Local Variables:
%%% mode: latex
%%% TeX-master: "../ClassicThesis"
%%% ispell-dictionary: "british" ***
%%% fill-column: 76 ***
%%% End:
