\chapter{Implementation}
\label{cha:implementation}
\epigraph{``The rule is, jam to-morrow and jam yesterday—but never jam to-day.''}{\textit{Through the Looking-Glass}\\\textsc{Lewis Carroll}}



Whereas the previous chapter concerned itself with the overall plan in the
abstract, this is where the actual experiment in the form of an
implementation is taking form.  It is not the purpose of the implementation
to fully realise the design described in the previous chapter. It is the
exclusive purpose of the implementation (a subset of the design) to either
validate or refute the hypotheses put forth in the introduction. This, and
nothing else. If it does less, you have posed questions you are not prepared
to answer; if it does more, you should be coding less or asking additional
questions.

The primary purpose of this chapter is to clearly communicate what has been
built, and how it works. This can, \eg include architectural diagrams,
software and hardware overviews and specifics.

If it illustrates core aspects, \eg the inner working of a particular
important algorithm or function, code segments are welcome in this chapter,
as long as they are short, to the point, well-commented and -formatted.  For
algorithms, pseudo code is often clearer than actual code, and for, \eg
\acs{API} examples the reverse holds true.  It is also a good idea to
provide the reader with a general overview of the structure of the code, as
well as how communication between various parts takes place.  As in the
previous chapter, I recommend using \ac{UML} for this purpose.  The complete
code (as well as your data) should be included separately with your report
in the form of a zip-file, USB-stick, or link to a repository.

Overall, the implementation is the computer scientist's equivalent of lab
equipment carefully arranged into an experimental setup, and just as the
validity of an experimental investigation will be judged in part on the
craftsmanship of the setup, so will the quality of your implementation. It
is therefore important to clearly communicate how your system works and how
it was built, so that the reader may have confidence in your evaluation and
conclusions.

%\begin{lstlisting}{float=b,language=Pascal,frame=tb,caption={A floating example (\texttt{listings} manual)},label=lst:useless}


\begin{lstlisting}[float, language=C, caption=A small C program, label=lst:smallcprogram]
#include <stdio.h>
#include <stdlib.h>

int main(int argc, char *argv[])
{
  FILE *fp;
  
  if (argc != 2) {
    fprintf(stderr, "Usage: %s <file>\n", argv[0]);
    exit(EXIT_FAILURE);
  }
  
  fp = fopen(argv[1], "r");
  if (fp == NULL) {
    perror(argv[1]);
    exit(EXIT_FAILURE);
  }
  
  /* Other code omitted */
  
  fclose(fp);
  exit(EXIT_SUCCESS);
}
\end{lstlisting}


%%% Local Variables:
%%% mode: latex
%%% TeX-master: "../ClassicThesis"
%%% ispell-dictionary: "british" ***
%%% fill-column: 76 ***
%%% End:
